\section{Conclusion}
Our purpose with this article was to discover the challenges that surround a Microservice Architecture development. That objective was accomplished through the development of a new Microservice-based tool that works besides already well-established services like Elasticsearch and Kibana, this way we can experience what are the difficulties that exist when developing services through a Microservice Architecture.

Through our development, we discovered tools like a pre-built eCommerce shop that saved us a lot of time by providing a basic but loaded with all the features we needed to feed our planned analysis service. As a given on a Microservice architecture, all the services are distributed on different containers, on our case we used Docker. 

The next step was setting up a connection between the two, to do this we used the ELK stack that provided us with a way to receive all the logs from the Shop through the Logstash and Elasticsearch as a way to index all the data being received.

And the final step was figuring out how to make a proper analysis, this problem was solved through the use of Python and some Elasticsearch functions. The result is provided to our users by our own front-end based on Vue with data from our Djangorest back-end.